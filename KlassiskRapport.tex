\documentclass[5p]{elsarticle}        % 5p gir 2 kolonner pr side. 1p gir 1 kolonne pr side.
\journal{Fagl\ae rer}
\usepackage[T1]{fontenc} 				% Vise norske tegn.
\usepackage[norsk]{babel}				% Tilpasning til norsk.
\usepackage[utf8]{inputenc}             % selv puttet inn for � kunne skrive norske tegn
\usepackage{graphicx}       				% For � inkludere figurer.
\usepackage{amsmath,amssymb} 				% Ekstra matematikkfunksjoner.
\usepackage{siunitx}					% M� inkluderes for blant annet � f� tilgang til kommandoen \SI (korrekte m�ltall med enheter)
	\sisetup{exponent-product = \cdot}      	% Prikk som multiplikasjonstegn (i steden for kryss).
 	\sisetup{output-decimal-marker  =  {,}} 	% Komma som desimalskilletegn (i steden for punktum).
 	\sisetup{separate-uncertainty = true}   	% Pluss-minus-form p� usikkerhet (i steden for parentes). 
\usepackage{booktabs}                     		% For � f� tilgang til finere linjer (til bruk i tabeller og slikt).
\usepackage[font=small,labelfont=bf]{caption}		% For justering av figurtekst og tabelltekst.
\usepackage{comment}                            % for � kunne kommentere i bolker (med \begin{comment} og \end{comment}
\usepackage[export]{adjustbox}            % for � kunne plassere figurer bedre, h�yre venstre justering
\usepackage[below]{placeins}                      %tillatter bruk av \FloatBarrier ([above], [below])
\usepackage{gensymb}                    % gj�r at man kan bruke \celsius og \degree

%\usepackage[section]{below}        %begrenser figuerer til sin section, kan bruker i stedet for \usepackage{placeins} 

% Denne setter navnet p� abstract til Sammendrag
\renewenvironment{abstract}{\global\setbox\absbox=\vbox\bgroup
\hsize=\textwidth\def\baselinestretch{1}%
\noindent\unskip\textbf{Sammendrag}
\par\medskip\noindent\unskip\ignorespaces}
{\egroup}


% Disse kommandoene kan gj�re det enklere for LaTeX � plassere figurer og tabeller der du �nsker.
\setcounter{totalnumber}{5}
\renewcommand{\textfraction}{0.05}
\renewcommand{\topfraction}{0.95}
\renewcommand{\bottomfraction}{0.95}
\renewcommand{\floatpagefraction}{0.35}

%%%%%%%%%%%%%%%%%%%%%%%%%%%%%%%%%%%%%%%%%%%%%%%%%%%%%%%%%%%%%%%%%%%%%%%%%
\begin{document}

\begin{frontmatter}


\title{...}

\author[fysikk]{H\aa kon Task{\'{e}}n\ }
\author[fysikk]{Paul Thrane}
\address[fysikk]{Institutt for fysikk, Norges Teknisk-Naturvitenskapelige Universitet, N-7491 Trondheim, Norway.}

\begin{abstract}


\end{abstract}

\end{frontmatter}


%%%%%%%%%%%%%%%%%%%%%%%%%%%%%%%%%%%%%%%%%%%%%%%%%%%%%%%%%%%%%%%%%%%%%%%%%
\section{Innledning}



%%%%%%%%%%%%%%%%%%%%%%%%%%%%%%%%%%%%%%%%%%%%%%%%%%%%%%%%%%%%%%%%%%%%%%%%%
\section{Teori}

%%%%%%%%%%%%%%%%%%%%%%%%%%%%%%%%%%%%%%%%%%%%%%%%%%%%%%%%%%%%%%%%%%%%%%%%%
\section{Metode}

%%%%%%%%%%%%%%%%%%%%%%%%%%%%%%%%%%%%%%%%%%%%%%%%%%%%%%%%%%%%%%%%%%%%%%%%%
\section{Resultat}

%\begin{figure}[h] 
%	\begin{center}
%		\includegraphics[width=.55\textwidth,center]{} 
%	\end{center}
%		\caption{} 
%		\label{} % Som med ligningen, er dette navnet vi refererer til.
%\end{figure}

%%%%%%%%%%%%%%%%%%%%%%%%%%%%%%%%%%%%%%%%%%%%%%%%%%%%%%%%%%%%%%%%%%%%%%%%%
\section{Diskusjon}

%%%%%%%%%%%%%%%%%%%%%%%%%%%%%%%%%%%%%%%%%%%%%%%%%%%%%%%%%%%%%%%%%%%%%%%%%
\section{Konklusjon}

%%%%%%%%%%%%%%%%%%%%%%%%%%%%%%%%%%%%%%%%%%%%%%%%%%%%%%%%%%%%%%%%%%%%%%%%%
\section*{Referanser}

\begin{thebibliography}{99}

\bibitem{macmichael}
Macmichael, D.B.A., Reay, D.A. Heat Pumps. Pergamon Press, 2nd Edition, 1988.

\bibitem{labheftet}
Naqvi, Kalbe Razi. Laboratorium i emnene TFY4165/FY1005 Termisk Fysikk. NTNU, V\aa ren 2015.

\bibitem{hemmer}
Hemmer, P.C. Termisk fysikk. Tapir Akademiske Forlag, 2. utgave, Trondheim 2002.

\bibitem{dupont}
DuPont. Thermodynamic Properties of HFC-134a (1,1,1,2-tetrafluoroetan). DuPont, 2004.


\bibitem{varmetapRor}
The Engineering Toolbox. Heat Loss of Uninsulated Copper Tubes. http://www.engineeringtoolbox.com/copper-pipe-heat-loss-d\_19.html (Mars 2015).

\bibitem{varmetapBotte}
The Engineering Toolbox. Heat Loss from Open Water Tanks. http://www.engineeringtoolbox.com/heat-loss-open-water-tanks-d\_286.html (Mars 2015).

\bibitem{vannInfo}
The Engineering Toolbox. Water - Thermal properties. http://www.engineeringtoolbox.com/water-thermal-properties-d\_162.html (Mars 2015).

\end{thebibliography}


\end{document}


% varmepumpa eller varmepumpen?
% trenger jeg � oppgi hvordan jeg kommer fram til delta noe?Ma